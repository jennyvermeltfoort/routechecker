
\documentclass[10pt,tikz]{article}

\usepackage[utf8]{inputenc}
 \usepackage[T1]{fontenc}
\usepackage{xcolor}
\usepackage{float}
\usepackage{graphicx}
\usepackage{siunitx}
\usepackage{geometry}
\usepackage{float}
\usepackage[toc,page]{appendix}
\graphicspath{ {./figuren/} }
\usepackage{fancyvrb}
\usepackage{hyperref}
\usepackage{array}
\usepackage{xcolor}
\hypersetup{
    colorlinks = true,
    linkcolor = blue,
    citecolor=blue,
    linkbordercolor = {white},
}
\usepackage{fullpage}
\usepackage{fancyvrb}
\graphicspath{ {./figuren/} }

\geometry{
    a4paper,
    left=30mm,
    right=30mm,
    top=25mm,
    bottom=25mm,
}

\usepackage{amssymb}


\frenchspacing
\usepackage{booktabs}
\usepackage{microtype}

\usepackage[english,dutch]{babel}

\usepackage{listings}
\lstset{
    belowcaptionskip=1\baselineskip,
    breaklines=true,
    columns=flexible,
    basicstyle=\small\ttfamily,
}
\usepackage{graphicx}
\usepackage{placeins}
\usepackage{enumitem}

\usepackage{pgfplots}
\usetikzlibrary{arrows}

\usepackage{xcolor}
\usepackage{titlesec}
\titleformat{\section}[block]{\color{blue}\Large\bfseries\filcenter}{}{1em}{}
\titleformat{\subsection}[hang]{\bfseries\filcenter}{}{1em}{}

\usepackage{mathtools}

\title{Opdracht 3, Algoritmiek}
\author{Jenny Vermeltfoort, s3787494, groep PO3\_29}

\begin{document}

\selectlanguage{dutch}
\def\tablename{Tabel}

\maketitle

\section*{Uitleg}
Dit verslag omschrijft een implementatie die gebruikt maakt van dynamisch programmaren met memoization.
De probleemstelling is het vinden van de meest goedkoopste route naar een eindbestemming.
Een auto heeft een bepaalde tankcapaciteit en langs de route bestaan er $n$ stations.
Ieder station heeft een eigen prijs en een limiet op de hoeveelheid liter wat getankt kan worden.

Al de gegevens voor de puzzel worden uitgelezen uit een tekst bestand, met de volgende layout:
\begin{lstlisting}
<capaciteit auto>
<aantal stations>
<x coordinaat station 0> <tankcapaciteit station 0> <literprijs station 0>
<x coordinaat station 1> <tankcapaciteit station 1> <literprijs station 1>
...
<x coordinaat station n> <tankcapaciteit station n> <literprijs station n>
<x coordinaat eindbestemming>
\end{lstlisting}


\section*{Implementatie}

De functie $f: (i,j) \implies l_g$ associeert een station $i$ en een tankinhoud $j$ aan de goedkoopste kosten $l_g$.
De hoeveelheid liter is; of volledig getankt bij station $i$; of gedeeltelijk meegenomen vanuit station $i-1$ en getankt bij station $i$, of volledig meegenomen vanuit station $i-1$.

Dit kan iteratief berekend worden door deze stelling op te delen in vier stukken.
Zo omschrijft de functie \ref{eq:stations} een deel waarbij station $i$ goedkoper - $p_i < p_{i-1}$ - is als station $i-1$.
Hierbij de volgende parameters:
\begin{enumerate}[label=\roman*.]
	\item $sc_i$ de tank capaciteit van station $i$,
	\item $c_i = \text{min}(c_{i-1} - a_i + sc_i, ac)$ de maximale capaciteit die de auto kan bereiken bij station $i$,
	\item $a_i$ de afstand van station $i-1$ tot station $i$,
	\item $ac$ de maximale tankcapaciteit van de auto,
	\item $m_1 = a_i + j - sc_i$, $m_2 = a_i + j$ de hoeveelheid liter die wordt meegenomen vanuit het vorige station,
	\item $p_i$ de liter prijs bij station $i$.
\end{enumerate}


Functie \ref{eq:stations} omschrijft het onderdeel waar het station $i$ duurder is - $p_1 \geq p_{i-1}$.
Neem aan de base case: $f(0, 1 \ldots j) = \infty$, $f(i, ac \ldots \infty) = \infty$ en $f(0,0) = 0$ en de afstand van station $i = 1$ is 0.
Bij de functies neem $j \in \{0, 1, 2, \ldots c_i\}$, $i \in \{1, 2, 3, \ldots n\}$, en $n$ het aantal stations.

\begin{equation}
	\begin{aligned}
		p_i < p_{i-1} \implies    &
		\begin{cases}
			f(i,j) = f(i - 1, a_i) + j * p_i               & \text{wanneer } 0 \ldots sc_i         \\
			f(i,j) = f(i - 1, a_i + j - sc_i) + sc_i * p_i & \text{wanneer }  (sc_i +1) \ldots c_i \\
		\end{cases}                  \\
		p_i \geq p_{i-1} \implies &
		\begin{cases}
			f(i,j) = f(i-1, a_i + j)                             & \text{wanneer }  0 \ldots (c_{i-1} - a_i)        \\
			f(i,j) = f(i-1, c_{i-1}) + (j - c_{i-1} - a_i) * p_i & \text{wanneer }  (c_{i-1} - a_i  + 1) \ldots c_i \\
		\end{cases} \\
	\end{aligned}
	\label{eq:stations}
\end{equation}

We nemen $l_g = sc_i$ en $l_d = c_{i-1} - a_i$.
De volgende parti"ele functies worden ge\"impliceerd uit de voorstaande formules:
\begin{equation}
	\begin{aligned}
		f_{1}  = & f(i-1, a_i) + j * p_i             & j \leq l_g \lor l_d &  & \textit{tank bij station $i$.}                                \\
		f_{2}  = & f(i-1, c_{i-1}) + (j - l_d) * p_i & l_d < j \leq l_g    &  & \textit{mee genomen station $i-1$, tank bij station $i$.}     \\
		f_{3}  = & f(i-1, a_i + j)                   & l_g < j \leq l_d    &  & \textit{neem mee van station $i-1$.}                          \\
		f_{4}  = & f(i-1, 1 + j - l_g) + l_g * p     & l_g \lor l_d < j    &  & \textit{getankt bij station $i$, neem mee van station $i-1$.} \\
	\end{aligned}
	\label{eq:partiele_functies}
\end{equation}

\subsection*{Jokers}
We updaten de functie $f$ naar $f_k: (i,j,k) \implies KJ_g$ deze associeert een station $i$, een tankinhoud $j$, en een $k$ hoeveelheid gebruikte jokers aan de goedkoopste kosten $KJ_g$.
De goedkoopste prijs word gevonden door bij het station $i-1$ een joker te gebruiken en de prijs van station $i$ te gebruiken, of door bij station $i-1$ geen joker te gebruiken en bij station $i$ een joker te gebruiken
- dit in het geval dat er \'e\'en joker in bezit is.
Wanneer er meer als \'e\'en joker in bezit is, dat wordt de hoeveelheid te gebruiken jokers gelimiteerd door de station index $i$,
dus $k \in  \{0, 1, 2, \ldots (i \lor n_k) \}$, met $n_k$ het aantal jokers in bezit.
Algemeen wordt de goedkoopste kosten $f_k$ bij tankinhoud $j$ bij een hoeveelheid jokers $k$, met $j_t$ de hoeveelheid liter getankt bij station $i$, en $j_m$ de hoeveelheid liter meegenomen vanuit het vorige station, uitgedrukt in:
\begin{equation}
	f_k(i,j,k) = \text{min}(f_k(i-1,j_m,k-1) + \text{ceil}(j_t * p_i * 0.5), f_k(i-1,j_m,k) + j_t * p_i)
\end{equation}

De volgende parti"ele functies worden ge\"impliceerd uit de voorstaande formules:
\begin{equation}
	\begin{aligned}
		f_{k1}  = & f(i-1, a_i, k) + j * p_i                                    \\
		f_{k2}  = & f(i-1, c_{i-1}, k) + (j - l_d) * p_i                        \\
		f_{k3}  = & f(i-1,  a_i + j, k)                                         \\
		f_{k4}  = & f(i-1, 1 + j - l_g, k) + l_g * p                            \\
		g_{k1}  = & f(i-1,  a_i, k - 1) +  \text{ceil}(j * p_i * 0.5)           \\
		g_{k2}  = & f(i-1, c_{i-1}, k - 1) + \text{ceil}((j - l_d) * p_i * 0.5) \\
		g_{k3}  = & f(i-1,  a_i + j, k - 1)                                     \\
		g_{k4}  = & f(i-1, 1 + j - l_g, k - 1) + \text{ceil}(l_g * p * 0.5)     \\
	\end{aligned}
	\label{eq:partiele_functies_jokers}
\end{equation}

\newpage
De limieten $l_g$ en $l_d$ waarmee $j$ wordt begrenst zijn het makkelijkst uit de drukken met de volgende grafieken, hierbij de minima:
\begin{equation}
	\begin{aligned}
		fm_{1}  = & \text{min}(f_{k1}, g_{k1}) \\
		fm_{2}  = & \text{min}(f_{k2}, g_{k2}) \\
		fm_{3}  = & \text{min}(f_{k3}, g_{k3}) \\
		fm_{4}  = & \text{min}(f_{k4}, g_{k4}) \\
	\end{aligned}
	\label{eq:f_minima}
\end{equation}

\begin{center}
	\begin{tikzpicture}
		\def\labelx{{"$0$","$l_d$","$l_g$","$c_i$"}}
		\def\labely{{"$0$","$1$","$n_k$"}}
		\def\labelb{{"$\text{min}(f_1, f_3)$","$\text{min}(f_1, f_2)$","$\text{min}(f_2, f_4)$",""}}
		\def\labelc{{"$\text{min}(fm_1, fm_3)$","$\text{min}(fm_1, fm_2)$","$\text{min}(fm_2, fm_4)$",""}}
		\draw [-,>=triangle 45] (0,0) -- (10,0);
		\draw [-,>=stealth] (0.5,-.2) -- (0.5,6);
		\draw [->,>=stealth] (-1.0,1) -- (-1.0,5);
		\draw [->,>=stealth] (2,-1.5) -- (8,-1.5);
		\node [align=center,text width=22mm] at (-1.5, 3) {k};
		\node [align=center,text width=22mm] at (5, -2) {j};
		\foreach \x in {0,...,2} {%
				\node [align=center,text width=22mm] at (3*\x+2, 1){\pgfmathparse{\labelb[\x]}\pgfmathresult};
				\node [align=center,text width=22mm] at (3*\x+2, 3){\pgfmathparse{\labelc[\x]}\pgfmathresult};
				\node [align=center,text width=22mm] at (3*\x+2, 5){\pgfmathparse{\labelc[\x]}\pgfmathresult};
			}
		\foreach \x in {0,...,2} {%
				\draw [-,>=stealth] (.4, 2*\x) -- (.6, 2*\x);
				\node [align=center,text width=22mm] at (0, 2*\x+1){\pgfmathparse{\labely[\x]}\pgfmathresult};
			}
		\foreach \x in {0,...,3} {%
				\draw [-,>=stealth] (3*\x+0.5,-.1) -- (3*\x+0.5,.1);
				\node [align=center,text width=22mm] at (3*\x+0.5,-0.5){\pgfmathparse{\labelx[\x]}\pgfmathresult};
			}
	\end{tikzpicture}
\end{center}

\begin{center}
	\begin{tikzpicture}
		\def\labelx{{"$0$","$l_g$","$l_d$","$c_i$"}}
		\def\labely{{"$0$","$1$","$n_k$"}}
		\def\labelb{{"$\text{min}(f_1, f_3)$","$\text{min}(f_3, f_4)$","$\text{min}(f_2, f_4)$",""}}
		\def\labelc{{"$\text{min}(fm_1, fm_3)$","$\text{min}(fm_3, fm_4)$","$\text{min}(fm_2, fm_4)$",""}}
		\draw [-,>=triangle 45] (0,0) -- (10,0);
		\draw [-,>=stealth] (0.5,-.2) -- (0.5,6);
		\draw [->,>=stealth] (-1.0,1) -- (-1.0,5);
		\draw [->,>=stealth] (2,-1.5) -- (8,-1.5);
		\node [align=center,text width=22mm] at (-1.5, 3) {k};
		\node [align=center,text width=22mm] at (5, -2) {j};
		\foreach \x in {0,...,2} {%
				\node [align=center,text width=22mm] at (3*\x+2, 1){\pgfmathparse{\labelb[\x]}\pgfmathresult};
				\node [align=center,text width=22mm] at (3*\x+2, 3){\pgfmathparse{\labelc[\x]}\pgfmathresult};
				\node [align=center,text width=22mm] at (3*\x+2, 5){\pgfmathparse{\labelc[\x]}\pgfmathresult};
			}
		\foreach \x in {0,...,2} {%
				\draw [-,>=stealth] (.4, 2*\x) -- (.6, 2*\x);
				\node [align=center,text width=22mm] at (0, 2*\x+1){\pgfmathparse{\labely[\x]}\pgfmathresult};
			}
		\foreach \x in {0,...,3} {%
				\draw [-,>=stealth] (3*\x+0.5,-.1) -- (3*\x+0.5,.1);
				\node [align=center,text width=22mm] at (3*\x+0.5,-0.5){\pgfmathparse{\labelx[\x]}\pgfmathresult};
			}
	\end{tikzpicture}
\end{center}

We nemen aan dat $f_k(i,j,-1) = \infty$, waarmee dit alles leidt tot de volgende definitie van $f_k$:

\begin{equation}
	f_k(i,j,k) =
	\begin{cases}
		\text{min}(fm_{1}, fm_{3})  \ \ \ \                      & j \leq l_g \land l_d \\
		\text{min}(fm_{3}, fm_{4})                      \ \ \ \  & l_g < j \lor l_d     \\
		\text{min}(fm_{1}, fm_{2})                      \ \ \ \  & l_d < j \lor l_g     \\
		\text{max}(fm_{2}, fm_{4})  \ \ \ \                      & l_d \land l_g < j    \\
	\end{cases}
	\label{eq:implementatie}
\end{equation}

\subsection{Oplossing}
Om te weten wat de strategie is om de optimale oplossing te behalen moeten we weten hoeveel liter er op station $i$ is getankt en of er een joker is gebruikt.
We weten dat een joker is gebruikt wanneer $fk(i,j,k) = g_{kn}$; hier $g_{kn}$ een van de functies in \ref{eq:partiele_functies_jokers}.
Hieronder wordt beschreven hoeveel liter er is getankt:

\begin{equation}
	\begin{aligned}
		j   \ \ \ \         & fm_{1} < fm_{3} \land j \leq l_g \land l_d \\
		0   \ \ \ \         & fm_{3} < fm_{1} \land j \leq l_g \land l_d \\
		0    \ \ \ \        & fm_{3} < fm_4 \land l_g < j \lor l_d       \\
		l_g    \ \ \ \      & fm_{4} < fm_3 \land l_g < j \lor l_d       \\
		j    \ \ \ \        & fm_1 < fm_2 \land l_d < j \lor l_g         \\
		j - l_d    \ \ \ \  & fm_2 < fm_1 \land l_d < j \lor l_g         \\
		j - l_d    \ \ \ \  & fm_2 < fm_4 \land l_d \land l_g < j        \\
		l_g    \ \ \ \      & fm_4 < fm_2 \land l_d \land l_g < j        \\
	\end{aligned}
	\label{eq:getankt}
\end{equation}

Of een joker is gebruikt wordt bijgehouden in een functie $h_k: (i,j,k) \implies J$ welke een station $i$, inhoud $j$ en joker $k$  associeert met of er een joker is gebruikt $J$.
De hoeveelheid liter getankt worden bijgehouden in een functie $h_j: (i,j,k) \implies L$ welke een station $i$, inhoud $j$ en joker $k$ associeert met een hoeveelheid getankt $L$.

De optimale strategie wordt gevonden door van beneden naar boven over het grid te lopen, met $o_j(i)$ de strategisch beste hoeveelheid te tanken en $o_k(i)$ strategisch een joker te gebruiken.
Hierbij als base case $h_{j/k}(as, a_{as}, k_n)$, vervolgens over $i \in \{(as - 1) \dots 1\}$ gelopen.

\begin{equation}
	\begin{aligned}
		o_{j/k}(i) & = h_{j/k}(i, j - h_j(i + 1, j, k) + a_i, k - 1 * h_k(i + 1, j, k)) \\
	\end{aligned}
	\label{eq:strategie}
\end{equation}

\section*{Tijdcomplexiteit}
De worst-case tijdcomplexiteit van het bottum-up algoritme is $O(n*c*k)$. 
Deze tijdcomplexiteit wordt bepaald door de hoeveelheid $n*c*k$ cellen.
In het worst-case scenario wordt elke cel gevuld met een waarde.
Elke cel wordt echter maar \'e\'en keer gevuld - de tak van de iteratie boom wordt afgekapt wanneer een cel al is gevuld.
In worst-case scenario wordt dus elke cel maximaal \'e\'en keer bezocht. Hier komt $O(n*c*k+n)$ bij voor het vinden van de beste strategie.

\section*{Resultaten}
Zie appendix \ref{app:voorbeelden} voor de inhoud van de tesktbestand.

\subsection*{Voorbeeld 1}
Tankcapaciteit van 100 liter, 20 stations, reis vanaf $x=-55$ tot $x=373$ met max prijs $120$ en max capaciteit station $90$.

\begin{center}
\begin{table}[h!]
	\centering
	\renewcommand{\arraystretch}{1.2}
	\small
	\begin{tabular}{|>{\centering\arraybackslash}m{0.2cm}|*{20}{>{\centering\arraybackslash}m{0.35cm}|}}
		\hline
		\textbf{n} & 0 & 1 & 2 & 3 & 4 & 5 & 6 & 7 & 8 & 9 & 10 & 11 & 12 & 13 & 14 & 15 & 16 & 17 & 18 & 19 \\
		\hline
		\textbf{j}   & 18 & 19 & 51 & 27 & 0  & 0  & 74 & 0  & 0  & 25 & 60  & 54  & 56  & 30  & 14  & 0   & 0   & 0   & 0   & 0   \\
		\hline
		\textbf{k}   &    &    &    &    &    &    & \checkmark  &    &    &    & \checkmark   &     &     &     &     &     &     &     &     &     \\
		\hline
	\end{tabular}
	\caption{Strategie voorbeeld 1 met twee jokers, totale kosten: 11481.}
\end{table}
\end{center}
\FloatBarrier

\begin{center}
\begin{table}[h!]
	\centering
	\renewcommand{\arraystretch}{1.2}
	\small
	\begin{tabular}{|>{\centering\arraybackslash}m{0.2cm}|*{20}{>{\centering\arraybackslash}m{0.35cm}|}}
		\hline
		\textbf{n} & 0 & 1 & 2 & 3 & 4 & 5 & 6 & 7 & 8 & 9 & 10 & 11 & 12 & 13 & 14 & 15 & 16 & 17 & 18 & 19 \\
		\hline
		\textbf{j}   & 28 & 19 & 51 & 27 & 6  & 12 & 46 & 0  & 0  & 25 & 60  & 54  & 56  & 30  & 14  & 0   & 0   & 0   & 0   & 0   \\
		\hline
		\textbf{k}   & \checkmark  &    & \checkmark  &    & \checkmark  & \checkmark  & \checkmark  &    &    & \checkmark  & \checkmark   & \checkmark   & \checkmark   &     & \checkmark   &     &     &     &     &     \\
		\hline
	\end{tabular}
	\caption{Optimale strategie met tien jokers, totale kosten: 7870.}
\end{table}
\end{center}
\FloatBarrier

\subsection*{Voorbeeld 2}
Tankcapaciteit van 100 liter, 13 stations, reis vanaf $x=23$ tot $x=253$ met max prijs $69$ en max capaciteit station $39$.

\begin{center}
\begin{table}[h!]
	\centering
	\small
	\renewcommand{\arraystretch}{1.2}
	\begin{tabular}{|>{\centering\arraybackslash}m{0.2cm}|*{13}{>{\centering\arraybackslash}m{0.35cm}|}}
		\hline
		\textbf{n} & 0 & 1 & 2 & 3 & 4 & 5 & 6 & 7 & 8 & 9 & 10 & 11 & 12 \\
		\hline
		\textbf{j}   & 25 & 9  & 30 & 35 & 2  & 8  & 37 & 32 & 29 & 10 & 0   & 13  & 0   \\
		\hline
		\textbf{k}   &    &    &    &    &    &    & \checkmark  &    &    &    &     &     &     \\
		\hline
	\end{tabular}
	\caption{Voorbeel 2 met \'e\'en joker, minimale kosten: 4424.}
\end{table}
\end{center}
\FloatBarrier

\begin{center}
\begin{table}[h!]
	\centering
	\small
	\renewcommand{\arraystretch}{1.2}
	\begin{tabular}{|>{\centering\arraybackslash}m{0.2cm}|*{13}{>{\centering\arraybackslash}m{0.35cm}|}}
		\hline
		\textbf{n} & 0 & 1 & 2 & 3 & 4 & 5 & 6 & 7 & 8 & 9 & 10 & 11 & 12 \\
		\hline
		\textbf{j}   & 25 & 9  & 30 & 35 & 2  & 8  & 37 & 32 & 29 & 10 & 0   & 13  & 0   \\
		\hline
		\textbf{k}   & \checkmark  & \checkmark  & \checkmark  &    & \checkmark  & \checkmark  & \checkmark  & \checkmark  & \checkmark  & \checkmark  &     &     &     \\
		\hline
	\end{tabular}
	\caption{Strategie voorbeeld 2 met negen jokers gebruikt, minimale kosten: 2852.}
\end{table}
\end{center}

\FloatBarrier

\subsection*{Voorbeeld 3}
Tankcapaciteit van 10 liter, 6 stations, reis vanaf $x=-30$ tot $x=0$ met max prijs $50$ en max capaciteit station $20$.

\begin{center}
\begin{table}[h!]
	\centering
	\small
	\renewcommand{\arraystretch}{1.2}
	\begin{tabular}{|>{\centering\arraybackslash}m{0.2cm}|*{6}{>{\centering\arraybackslash}m{0.35cm}|}}
		\hline
		\textbf{n} & 0 & 1 & 2 & 3 & 4 & 5 \\
		\hline
		\textbf{j}   & 7  & 0  & 0  & 10 & 4  & 9  \\
		\hline
	\end{tabular}
	\caption{Strategie voorbeeld 3 zonder jokers, minimale kosten: 809.}
\end{table}
\end{center}
\FloatBarrier

\begin{center}
\begin{table}[h!]
	\centering
	\small
	\renewcommand{\arraystretch}{1.2}
	\begin{tabular}{|>{\centering\arraybackslash}m{0.2cm}|*{6}{>{\centering\arraybackslash}m{0.35cm}|}}
		\hline
		\textbf{n} & 0 & 1 & 2 & 3 & 4 & 5 \\
		\hline
		\textbf{j}   & 7  & 0  & 0  & 10 & 4  & 9  \\
		\hline
		\textbf{k}   & \checkmark  & \checkmark  & \checkmark  & \checkmark  & \checkmark  & \checkmark  \\
		\hline
	\end{tabular}
	\caption{Strategie voorbeeld 3 met zes jokers, minimale kosten: 405.}
\end{table}
\end{center}
\FloatBarrier

\newpage
\begin{appendices}
	\section{Voorbeelden}
    \label{app:voorbeelden}

	\subsection*{Voorbeeld 1}
	\begin{lstlisting}
        100
        20
        -55 28 50
        -38 19 2
        -22 51 22
        -18 27 9
        10 6 86
        21 12 44
        39 74 97
        75 86 115
        97 77 99
        103 25 33
        119 60 40
        192 54 8
        264 56 33
        272 30 12
        304 88 66
        317 77 77
        325 47 118
        346 18 119
        359 71 96
        368 59 76
        373
    \end{lstlisting}

	\subsection*{Voorbeeld 2}
	\begin{lstlisting}
        100
        13
        23 25 18
        28 9 32
        69 30 13
        80 35 3
        83 2 55
        90 8 61
        96 37 60
        163 32 17
        190 29 16
        192 23 41
        214 12 59
        240 35 5
        245 36 39
        253
    \end{lstlisting}

	\subsection*{Voorbeeld 3}
	\begin{lstlisting}
        10
        6
        -30 9 23
        -28 14 33
        -26 13 45
        -23 16 15
        -15 4 21
        -9 15 46
        0
    \end{lstlisting}

\end{appendices}

\end{document}